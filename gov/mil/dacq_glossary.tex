% !TEX encoding = UTF-8 Unicode

%Taken from Defense Acquisition Acronyms And Terms

%% A

\newglossaryentry{gov.mil.dacq.acceptance}{name=acceptance,
description={The act of an authorized representative of the government by which the government, for itself or as agent of another, assumes ownership of existing identified supplies tendered, or approves specific services rendered, as partial or complete performance of the contract on the part of the contractor. \protect \maybeCite{DoDAcqAcroTerms-2011}
}}

\newglossaryentry{gov.mil.dacq.accessibility}{name=accessibility,
description={A measure of the relative ease of admission to the various areas of an item for operation or maintenance. \protect \maybeCite{DoDAcqAcroTerms-2011}
}}

\newglossaryentry{gov.mil.dacq.accountsPayable}{name=accounts payable,
description={Amounts owed on open accounts, e.g., materials and services received, wages earned, and fringe benefits unpaid. \protect \maybeCite{DoDAcqAcroTerms-2011}
}}

\newglossaryentry{gov.mil.dacq.accountsReceivable}{name=accounts receivable,
description={Amounts due from debtors on open accounts; under-appropriated funds, amounts due from debtors for reimbursements earned or for appropriation refunds due. \protect \maybeCite{DoDAcqAcroTerms-2011}
}}

\newglossaryentry{gov.mil.dacq.accrualAccounting}{name=accrual accounting,
description={The basis of accounting whereby revenue is recognized when it is realized and expenses are recognized when incurred, without regard to time of receipt or payment of cash. \protect \maybeCite{DoDAcqAcroTerms-2011}
}}

\newglossaryentry{gov.mil.dacq.achievedAvailability}{name=achieved availability ($\text{A}_{\text{A}}$),
text={achieved availability},
description={Availability of a system with respect to operating time and both corrective and preventive maintenance. It ignores \glsname{gov.mil.dacq.meanLogisticsDelayTime} and may be calculated as \glsname{gov.mil.dacq.meanTimeBetweenMaintenance} divided by the sum of \acrshort{gov.mil.dacq.mtbm} and \glsname{gov.mil.dacq.meanMaintenanceTime}, that is, $\text{\acrshort{gov.mil.dacq.aa}} = \text{\acrshort{gov.mil.dacq.mtbm}}/(\text{\acrshort{gov.mil.dacq.mtbm}} + \text{\acrshort{gov.mil.dacq.mmt}})$. See \glsname{gov.mil.dacq.meanTimeBetweenMaintenance}, \glsname{gov.mil.dacq.meanLogisticsDelayTime}, and \glsname{gov.mil.dacq.meanMaintenanceTime}. \protect \maybeCite{DoDAcqAcroTerms-2011}
}}

\newglossaryentry{gov.mil.dacq.acquisition}{name=acquisition,
description={The conceptualization, initiation, design, development, testing, contracting, production, deployment, Logistics Support (LS), modification, and disposal of weapons and other systems, supplies, or services (including construction) to satisfy DoD needs, intended for use in, or in support of, military missions. \protect \maybeCite{DoDAcqAcroTerms-2011}
}}


%% C

\newglossaryentry{gov.mil.dacq.commerciallyAvailableOffTheShelf}{name=commercially available off-the-shelf (COTS),
text=commercially available off-the-shelf,
description={A \glsname{gov.mil.dacq.commercialItem} sold in substantial quantities in the commercial marketplace and offered to the government under a contract or subcontract at any tier, without modification, in the same form in which it was sold in the marketplace. This definition does not include bulk cargo such as agricultural products or petroleum.  \protect \maybeCite{DoDAcqAcroTerms-2011}; FAR, Subpart 2.101.
}}

\newglossaryentry{gov.mil.dacq.commercialItem}{name=commercial item (CI),
text=commercial item,
description={Any item, other than real property, that is of a type customarily used for nongovernmental purposes and that has been sold, leased, or licensed to the general public; or has been offered for sale, lease, or license to the general public; or any item evolved through advances in technology or performance and that is not yet available in the commercial marketplace but will be available in the commercial marketplace in time to satisfy the delivery requirements under a government solicitation. Also included are services in support of a CI of a type offered and sold competitively in substantial quantities in the commercial marketplace based on established catalog or market prices for specific tasks under standard commercial terms and conditions; this does not include services sold based on hourly rates without an established catalog or market price for a specified service. \protect \maybeCite{DoDAcqAcroTerms-2011}; FAR, Subpart 2.101.
}}


%% D

\newglossaryentry{gov.mil.dacq.derivedRequirement}{name=derived requirement,
description={These arise from constraints, consideration of issues implied but not explicitly stated in the requirements baseline, factors introduced by the selected architecture, Information Assurance (IA) requirements and the design. Derived requirements are definitized through requirements analysis as part of the overall Systems Engineering Process (SEP) and are part of the allocated baseline. \protect \maybeCite{DoDAcqAcroTerms-2011}
}}



%% L

\newglossaryentry{gov.mil.dacq.liveFireTestAndEvaluation}{name=Live Fire Test and Evaluation (LFT\&E),
text=Live Fire Test and Evaluation,
description={A test process that provides a timely assessment of the vulnerability and/or lethality of a conventional weapon or conventional weapon system as it progresses through its design and development. LFT\&E is a statutory requirement (Title 10 U.S.C. \S 2366) for covered systems, major munitions programs, missile programs, or product improvements to a covered system, major munitions programs, or missile programs before they can proceed beyond \glsname{gov.mil.dacq.lowRateInitialProduction}. See Covered System. \protect \maybeCite{DoDAcqAcroTerms-2011}
}}

\newglossaryentry{gov.mil.dacq.lineReplaceableUnit}{name=line replaceable unit (LRU),
text=Line Replaceable Unit,
description={An essential support item removed and replaced at field level to restore an end item to an operationally ready condition. (Also called Weapon Replacement Assembly (WRA) and Module Replaceable Unit.) \protect \maybeCite{DoDAcqAcroTerms-2011}
}}

\newglossaryentry{gov.mil.dacq.lowRateInitialProduction}{name=low-rate initial production (LRIP),
text=low-rate initial production,
description={1.) The first effort of the Production and Deployment (P\&D) phase. This effort is intended to result in completion of manufacturing development in order to ensure adequate and efficient manufacturing capability and to produce the minimum quantity necessary to provide production or production-representative articles for Initial Operational Test and Evaluation (IOT\&E); establish an initial production base for the system; and permit an orderly increase in the production rate for the system, sufficient to lead to Full-Rate Production (FRP) upon successful completion of operational (and live-fire, where applicable) testing. 2.) At Milestone B, the Milestone Decision Authority (MDA) determines the LRIP quantity for Major Defense Acquisition Programs (MDAPs) and major systems. The LRIP quantity for an MDAP (with rationale for quantities exceeding 10 percent of the total production quantity documented in the acquisition strategy) shall be included in the first Selected Acquisition Report (SAR) after its determination. The LRIP quantity shall not be less than one unit. The Director, Operational Test and Evaluation (DOT\&E), following consultation with the Program Manager (PM), determines the number of production or production-representative test articles required for Live-Fire Test and Evaluation (LFT\&E) and IOT\&E of programs on the Office of the Secretary of Defense (OSD) Test and Evaluation (T\&E) Oversight List. For a system that is not on the Oversight List, the Operational Test Agency (OTA), following consultation with the PM, shall determine the number of test articles required for IOT\&E. \protect \maybeCite{DoDAcqAcroTerms-2011}
}}







%% M

\newglossaryentry{gov.mil.dacq.meanLogisticsDelayTime}{name={mean logistics delay time (MLDT)},
text={mean logistics delay time},
description={Indicator of the average time a system is awaiting maintenance and generally includes time for locating parts and tools; locating, setting up, or calibrating test equipment; dispatching personnel; reviewing technical manuals; complying with supply procedures; and awaiting transportation. The MLDT is largely dependent upon the Logistics Support (LS) structure and environment. \protect \maybeCite{DoDAcqAcroTerms-2011}
}}

\newglossaryentry{gov.mil.dacq.meanMaintenanceTime}{name={mean maintenance time (MMT)},
text={mean maintenance time},
description={A measure of item maintainability taking into account both preventive and corrective maintenance. Calculated by adding the preventive and corrective maintenance time and dividing by the sum of scheduled and unscheduled maintenance events during a stated period of time. \protect \maybeCite{DoDAcqAcroTerms-2011}
}}

\newglossaryentry{gov.mil.dacq.meanTimeBetweenMaintenance}{name={mean time between maintenance (MTBM)},
text={mean time between maintenance},
description={A measure of reliability that represents the average time between all maintenance actions, both corrective and preventive. \protect \maybeCite{DoDAcqAcroTerms-2011}
}}


\newglossaryentry{gov.mil.dacq.moduleReplaceableUnit}{name=module replaceable unit,
description={ See \glstext{gov.mil.dacq.lineReplaceableUnit}. \protect \maybeCite{DoDAcqAcroTerms-2011}
}}


%% N

\newglossaryentry{gov.mil.dacq.nonDevelopmentalItem}{name={non-developmental item (NDI)},
text={non-developmental item},
description={1.) An NDI is any previously developed item of supply used exclusively for government purposes by a federal agency, a state or local government, or a foreign government with which the United States has a mutual defense cooperation agreement. 2.) Any item described in item 1 that requires only minor modifications or modifications of the type customarily available in the commercial marketplace in order to meet the requirements of the procuring department or agency. 3.) Any item of supply being produced that does not meet the requirements of items 1 or 2 solely because the item is not yet in use. (FAR 2.101) \protect \maybeCite{DoDAcqAcroTerms-2011}},
see=[See also]{gov.mil.dacq.commerciallyAvailableOffTheShelf}
}


%% P

\newglossaryentry{gov.mil.dacq.preventiveMaintenance}{name={preventive maintenance},
description={All actions performed in an attempt to retain an item in a specified condition by providing systematic inspection, detection, and prevention of incipient failures. \protect \maybeCite{DoDAcqAcroTerms-2011}
}}



%% R

\newglossaryentry{gov.mil.dacq.reliability}{name={reliability},
description={The ability of a system and its parts to perform their mission without failure, degradation, or demand on the support system under a prescribed set of conditions. See Mean Time Between Failure (MTBF) and \glsname{gov.mil.dacq.meanTimeBetweenMaintenance}. \protect \maybeCite{DoDAcqAcroTerms-2011}
}}


\newglossaryentry{gov.mil.dacq.robustDesign}{name={robust design},
description={The design of a system such that its performance is insensitive to variations in manufacturing tolerances, or its operational environment (including maintenance, transportation, and storage), or to component drift as a result of aging. \protect \maybeCite{DoDAcqAcroTerms-2011}
}}


%% S

\newglossaryentry{gov.mil.dacq.scheduledMaintenance}{name={scheduled maintenance},
description={Preventive maintenance performed at prescribed points in the item’s life. \protect \maybeCite{DoDAcqAcroTerms-2011}
}}

\newglossaryentry{gov.mil.dacq.supportEquipment}{name={support equipment (SE)},
text={support equipment},
description={All equipment (mobile or fixed) required to support the Operation and Maintenance (O\&M) of a materiel system. This includes associated multi-use support items, ground-handling and maintenance equipment, tools, meteorology and calibration equipment, and manual/automatic test equipment. It includes the acquisition of Logistics Support (LS) for the support equipment itself. One of the traditional LS elements. \protect \maybeCite{DoDAcqAcroTerms-2011}
}}


%% W

\newglossaryentry{gov.mil.dacq.weaponReplacementAssembly}{name=weapon replacement assembly (WRA),
text=weapon replacement assembly,
description={See \glstext{gov.mil.dacq.lineReplaceableUnit}. \protect \maybeCite{DoDAcqAcroTerms-2011}
}}
