% !TEX encoding = UTF-8 Unicode

% Taken from MIL-HDBK-1908, Definitions of Human Factors Terms

% A

\newglossaryentry{gov.mil.hf.anthropometricDimensions}{name=anthropometric dimensions,
description={Measured dimensions that describe the size and shape of the human body. These dimensions are often presented in the form of summary statistics that
describe the range of body dimensions that are observed in a population. Anthropometric terms are defined by MIL-HDBK-743. \protect \maybeCite{MILHDBK-1908B-2004}
}}

\newglossaryentry{gov.mil.hf.anthropometry}{name=anthropometry,
description={The scientific measurement and collection of data about human physical characteristics and the application (engineering anthropometry) of these data to the design and evaluation of systems, equipment, and facilities. \protect \maybeCite{MILHDBK-1908B-2004}
}}



% B



% C

\newglossaryentry{gov.mil.hf.commonHandTool}{name=common hand tool,
description={A tool found in common usage or applicable to a variety of operations or to a single operation on a variety of material. Screwdrivers, hammers, and wrenches are examples of common hand tools. \protect \maybeCite{MILHDBK-1908B-2004}
}}

\newglossaryentry{gov.mil.hf.commonPart}{name=common part,
description={A part or component which is generic because (a) equivalent parts are available from more than one manufacturer and (b) it is not designed or intended for exclusive use in or by a single system or piece of equipment. \protect \maybeCite{MILHDBK-1908B-2004}
}}

\newglossaryentry{gov.mil.hf.commonTool}{name=common tool,
description={A tool, routinely found in the tool supply of maintenance organizations for a similar class of system or equipment, which is generic because it is available from more than one manufacturer, and is not designed or intended for exclusive use on or with a single system or piece of equipment. \protect \maybeCite{MILHDBK-1908B-2004}
}}

\newglossaryentry{gov.mil.hf.components}{name=components,
description={Those constituent materials, parts, assemblies, and subassemblies that make up a piece of equipment or a unit. See also \glsname{hf:unit} \protect \maybeCite{MILHDBK-1908B-2004}
}}

\newglossaryentry{gov.mil.hf.compound number}{name=compound number,
description={A quantity involving different units of measure, for example, 3 feet 4 inches or 10 pounds 5 ounces. \protect \maybeCite{MILHDBK-1908B-2004}
}}

\newglossaryentry{gov.mil.hf.confirm}{name=confirm,
description={When used relative to test and evaluation, ``confirm'' implies a qualitative test that requires comparison of test results to an applicable requirement(s). See \glsname{hf:demonstration}. \protect \maybeCite{MILHDBK-1908B-2004}
}}

\newglossaryentry{gov.mil.hf.connector}{name=connector,
description={A piece of hardware that joins or attaches lines or cables to other lines or cables or to items of equipment. The term is used rather loosely to refer to either of two parts that mate with each other or to the plug that mates with a receptacle. \protect \maybeCite{MILHDBK-1908B-2004}
}}

\newglossaryentry{gov.mil.hf.contrast}{name=contrast,
description={See \glsname{hf:luminanceContrast}. \protect \maybeCite{MILHDBK-1908B-2004}
}}

\newglossaryentry{gov.mil.hf.control}{name=control,
description={A mechanism used to regulate or guide the operation of a machine, equipment component, subsystem, or system. \protect \maybeCite{MILHDBK-1908B-2004}
}}





% M

\newglossaryentry{gov.mil.hf.maintenance}{name=maintenance,
description={All actions necessary for retaining material in (or restoring it to) a serviceable condition. Maintenance includes servicing, repair, modification, modernization, overhaul, inspection, condition determination, corrosion control, and initial provisioning of support items. \protect \maybeCite{MILHDBK-1908B-2004}
}}
