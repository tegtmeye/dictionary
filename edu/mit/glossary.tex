% !TEX encoding = UTF-8 Unicode


%% C

\newglossaryentry{edu.mit.capacity}{name=capacity [\protect \si{\protect\ampere\protect\hour}],
text={capacity}
description={The coulometric capacity, the total Amp-hours available when the battery is discharged at a certain discharge current (specified as a \glslink{edu.mit.cAndERate}{C-rate}) from 100 percent \glstext{edu.mit.stateOfCharge} to the \glstext{edu.mit.cutOffVoltage}. Capacity is calculated by multiplying the discharge current (in Amps) by the discharge time (in hours) and decreases with increasing \glslink{edu.mit.cAndERate}{C-rate}.\protect \maybeCite{edu.mit.AGuideToUnderstandingBatterySpecifications:2008}
}}

\newglossaryentry{edu.mit.chargeVoltage}{name=charge voltage,
description={The voltage that the battery is charged to when charged to full capacity. Charging schemes generally consist of a constant current charging until the battery voltage reaching the charge voltage, then constant voltage charging, allowing the charge current to taper until it is very small. \protect \maybeCite{edu.mit.AGuideToUnderstandingBatterySpecifications:2008}
}}


\newglossaryentry{edu.mit.cAndERate}{name=C- and E-rates,
description={In describing batteries, discharge current is often expressed as a \glslink{edu.mit.cAndERate}{C-rate} in order to normalize against battery capacity, which is often very different between batteries. A \glslink{edu.mit.cAndERate}{C-rate} is a measure of the rate at which a battery is discharged relative to its maximum capacity. A 1C rate means that the discharge current will discharge the entire battery in 1 hour. For a battery with a capacity of 100 Amp-hrs, this equates to a discharge current of 100 Amps. A 5C rate for this battery would be 500 Amps, and a C/2 rate would be 50 Amps. Similarly, an E-rate describes the discharge power. A 1E rate is the discharge power to discharge the entire battery in 1 hour. \protect \maybeCite{edu.mit.AGuideToUnderstandingBatterySpecifications:2008}
}}






\newglossaryentry{edu.mit.cutOffVoltage}{name=cut-off voltage,
description={The minimum allowable voltage. It is this voltage that generally defines the ``empty'' state of the battery.\protect \maybeCite{edu.mit.AGuideToUnderstandingBatterySpecifications:2008}
}}

\newglossaryentry{edu.mit.cycleLife}{name=cycle life [number for a specific \glssymbol{edu.mit.depthOfDischarge}],
text={cycle life}
description={The number of discharge-charge cycles the battery can experience before it fails to meet specific performance criteria. Cycle life is estimated for specific charge and discharge conditions. The actual operating life of the battery is affected by the rate and depth of cycles and by other conditions such as temperature and humidity. The higher the \glsname{edu.mit.depthOfDischarge}, the lower the cycle life. \protect \maybeCite{edu.mit.AGuideToUnderstandingBatterySpecifications:2008}
}}



%% D

\newglossaryentry{edu.mit.depthOfDischarge}{name=depth of discharge (DOD) [\%],
text={depth of discharge},
description={The percentage of battery capacity that has been discharged expressed as a percentage of maximum capacity. A discharge to at least \protect\SI{80}{\protect\percent} DOD is referred to as a \emph{deep discharge}. \protect \maybeCite{edu.mit.AGuideToUnderstandingBatterySpecifications:2008}
}}


%% E

\newglossaryentry{edu.mit.energy}{name=energy [\protect\si{\protect\watt\protect\hour}],
text={energy}
description={The ``energy capacity'' of the battery, the total Watt-hours available when the battery is discharged at a certain discharge current (specified as a \glslink{edu.mit.cAndERate}{C-rate}) from 100 percent \glstext{edu.mit.stateOfCharge} to the \glstext{edu.mit.cutOffVoltage}. Energy is calculated by multiplying the discharge power (in Watts) by the discharge time (in hours). Like \glstext{edu.mit.capacity}, energy decreases with increasing \glslink{edu.mit.cAndERate}{C-rate}. \protect \maybeCite{edu.mit.AGuideToUnderstandingBatterySpecifications:2008}
}}

\newglossaryentry{edu.mit.energyDensity}{name=energy density [\protect\si{\protect\watt\protect\per\protect\liter}],
text={energy density}
description={The nominal battery energy per unit volume, sometimes referred to as the volumetric energy density. Specific energy is a characteristic of the battery chemistry and packaging. Along with the energy consumption of the vehicle, it determines the battery size required to achieve a given electric range.\protect \maybeCite{edu.mit.AGuideToUnderstandingBatterySpecifications:2008}
}}


%% F

\newglossaryentry{edu.mit.floatVoltage}{name=float voltage,
description={The voltage at which the battery is maintained after being charge to 100 percent \glsname{edu.mit.stateOfCharge} to maintain that capacity by compensating for self-discharge of the battery. \protect \maybeCite{edu.mit.AGuideToUnderstandingBatterySpecifications:2008}
}}


%% I

\newglossaryentry{edu.mit.internalResistance}{name=internal resistance,
description={The resistance within the battery, generally different for charging and discharging, also dependent on the battery \glsname{edu.mit.stateOfCharge}. As internal resistance increases, the battery efficiency decreases and thermal stability is reduced as more of the charging energy is converted into heat. \protect \maybeCite{edu.mit.AGuideToUnderstandingBatterySpecifications:2008}
}}


%% M

\newglossaryentry{edu.mit.maximumContinuousDischargeCurrent}{name=maximum continuous discharge current,
description={The maximum current at which the battery can be discharged continuously. This limit is usually defined by the battery manufacturer in order to prevent excessive discharge rates that would damage the battery or reduce its capacity. Along with the maximum continuous power of the motor, this defines the top sustainable speed and acceleration of the vehicle. \protect \maybeCite{edu.mit.AGuideToUnderstandingBatterySpecifications:2008}
}}


\newglossaryentry{edu.mit.maximumInternalResistance}{name=(maximum) internal resistance,
description={The resistance within the battery, generally different for charging and discharging.\protect \maybeCite{edu.mit.AGuideToUnderstandingBatterySpecifications:2008}
}}


\newglossaryentry{edu.mit.maximumThirtySecDischargePulseCurrent}{name=maximum 30 second discharge pulse current,
description={The maximum current at which the battery can be discharged for pulses of up to 30 seconds. This limit is usually defined by the battery manufacturer in order to prevent excessive discharge rates that would damage the battery or reduce its capacity. Along with the peak power of the electric motor, this defines the acceleration performance (0-60 mph time) of the vehicle. \protect \maybeCite{edu.mit.AGuideToUnderstandingBatterySpecifications:2008}
}}



%% N

\newglossaryentry{edu.mit.nominalCapacity}{name=nominal capacity [\protect\si{\protect\ampere\protect\hour}],
text={nominal capacity},
description={See \glstext{edu.mit.capacity}. \protect \maybeCite{edu.mit.AGuideToUnderstandingBatterySpecifications:2008}
}}

\newglossaryentry{edu.mit.nominalVoltage}{name=nominal voltage [\protect\si{\protect\volt}],
text={nominal voltage},
description={The reported or reference voltage of the battery, also sometimes thought of as the ``normal'' voltage of the battery.\protect \maybeCite{edu.mit.AGuideToUnderstandingBatterySpecifications:2008}
}}


%% O

\newglossaryentry{edu.mit.openCircuitVoltage}{name=open-circuit voltage [\protect\si{\protect\volt}],
text={open-circuit voltage},
description={The voltage between the battery terminals with no load applied. The open-circuit voltage depends on the battery \glsname{edu.mit.stateOfCharge}, increasing with \glssymbol{edu.mit.stateOfCharge}. \protect \maybeCite{edu.mit.AGuideToUnderstandingBatterySpecifications:2008}
}}

\newglossaryentry{edu.mit.powerDensity}{name=power density [\protect\si{\protect\watt\protect\per\protect\liter}],
text={power density},
description={The maximum available power per unit volume. Specific power is a characteristic of the battery chemistry and packaging. It determines the battery size required to achieve a given performance target. \protect \maybeCite{edu.mit.AGuideToUnderstandingBatterySpecifications:2008}
}}


%% R

\newglossaryentry{edu.mit.recommendedChargeCurrent}{name=(recommended) charge current,
description={The ideal current at which the battery is initially charged (to roughly 70 percent \glsname{edu.mit.stateOfCharge}) under constant charging scheme before transitioning into constant voltage charging. \protect \maybeCite{edu.mit.AGuideToUnderstandingBatterySpecifications:2008}
}}



%% S

\newglossaryentry{edu.mit.specificEnergy}{name=specific energy [\protect\si{\protect\watt\protect\hour\protect\per\protect\kilo\protect\gram}],
text={specific energy},
description={The maximum available power per unit mass. Specific power is a characteristic of the battery chemistry and packaging. It determines the battery weight required to achieve a given performance target. \protect \maybeCite{edu.mit.AGuideToUnderstandingBatterySpecifications:2008}
}}

\newglossaryentry{edu.mit.specificPower}{name=specific power [\protect\si{\protect\watt\protect\per\protect\kilo\protect\gram}],
text={specific power},
description={The nominal battery energy per unit mass, sometimes referred to as the gravimetric energy density. Specific energy is a characteristic of the battery chemistry and packaging. Along with the energy consumption of the vehicle, it determines the battery weight required to achieve a given electric range. \protect \maybeCite{edu.mit.AGuideToUnderstandingBatterySpecifications:2008}
}}

\newglossaryentry{edu.mit.stateOfCharge}{name=state of charge (SOC) [\%],
text={state of charge},
description={An expression of the present battery capacity as a percentage of maximum capacity. SOC is generally calculated using current integration to determine the change in battery capacity over time. \protect \maybeCite{edu.mit.AGuideToUnderstandingBatterySpecifications:2008}
}}


%% T

\newglossaryentry{edu.mit.terminalVoltage}{name=terminal voltage [\protect\si{\protect\volt}],
text={terminal voltage},
description={The voltage between the battery terminals with load applied. Terminal voltage varies with \glsname{edu.mit.stateOfCharge} and discharge/charge current. \protect \maybeCite{edu.mit.AGuideToUnderstandingBatterySpecifications:2008}
}}

